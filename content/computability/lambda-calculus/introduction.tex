% Part: computability
% Chapter: lambda-calculus
% Section: introduction

\documentclass[../../../include/open-logic-section]{subfiles}

\begin{document}

\olfileid{cmp}{lam}{int}
\olsection{Introduction}

The lambda calculus was originally designed by Alonzo Church in the
early 1930s as a basis for constructive logic, and \emph{not} as a
model of the computable functions. But soon after the Turing
computable functions, the recursive functions, and the general
recursive functions were shown to be equivalent, lambda computability
was added to the list. The fact that this initially came as a small
surprise makes the characterization all the more interesting.

Lambda notation is a convenient way of referring to a function
directly by a symbolic expression which defines it, instead of
defining a name for it. Instead of saying ``let $f$ be the function
defined by $f(x) = x + 3$,'' one can say, ``let $f$ be the function
$\lambd[x][(x + 3)]$.''  In other words, $\lambd[x][(x+3)]$ is just a
\emph{name} for the function that adds three to its argument. In this
expression, $x$ is a dummy variable, or a placeholder: the same
function can just as well be denoted by $\lambd[y][(y + 3)]$. The
notation works even with other parameters around. For example, suppose
$g(x, y)$ is a function of two variables, and $k$ is a natural
number. Then $\lambd[x][g(x,k)]$ is the function which maps any~$x$
to~$g(x, k)$.

This way of defining a function from a symbolic expression is known as
\emph{lambda abstraction}. The flip side of lambda abstraction is
\emph{application}: assuming one has a function $f$ (say, defined on
the natural numbers), one can \emph{apply} it to any value, like 2. In
conventional notation, of course, we write~$f(2)$ for the result.

What happens when you combine lambda abstraction with application?
Then the resulting expression can be simplified, by ``plugging'' the
applicand in for the abstracted variable. For example,
\[
(\lambd[x][(x + 3)])(2)
\]
can be simplified to~$2 + 3$.

Up to this point, we have done nothing but introduce new notations for
conventional notions. The lambda calculus, however, represents a more
radical departure from the set-theoretic viewpoint. In this framework:
\begin{enumerate}
\item Everything denotes a function.
\item Functions can be defined using lambda abstraction.
\item Anything can be applied to anything else.
\end{enumerate}
For example, if $F$ is a term in the lambda calculus, $F(F)$ is always
assumed to be meaningful. This liberal framework is known as the
\emph{untyped} lambda calculus, where ``untyped'' means ``no
restriction on what can be applied to what.''

\begin{digress}
There is also a \emph{typed} lambda calculus, which is an important
variation on the untyped version. Although in many ways the typed
lambda calculus is similar to the untyped one, it is much easier to
reconcile with a classical set-theoretic framework, and has some very
different properties.

Research on the lambda calculus has proved to be central in
theoretical computer science, and in the design of programming
languages. LISP, designed by John McCarthy in the 1950s, is an early
example of a language that was influenced by these ideas.
\end{digress}

\end{document}
