% Part:sets-functions-relations
% Chapter: sets
% Section: non-enumerability

\documentclass[../../../include/open-logic-section]{subfiles}

\begin{document}

\olfileid{sfr}{siz}{nen}

\olsection{\printtoken{S}{nonenumerable} Sets}

Some sets, such as the set $\Nat$ of natural numbers, are infinite. So
far we've seen examples of infinite sets which were all
!!{enumerable}. However, there are also infinite sets which do not
have this property. Such sets are called \emph{!!{nonenumerable}}.

Cantor's method of diagonalization shows a set to be !!{nonenumerable}
via a reductio proof. We start with the assumption that the set is
!!{enumerable}, and show that a contradiction results from this
assumption. Our first example is the set~$\Bin^\omega$ of all
infinite, non-gappy sequences of $0$'s and $1$'s.

\begin{thm}
\ollabel{thm-nonenum-bin-omega}
$\Bin^\omega$~is !!{nonenumerable}.
\end{thm}

\begin{proof}
Suppose, for reductio, that $\Bin^\omega$ is !!{enumerable}, so that
there is a list $s_{1}$, $s_{2}$, $s_{3}$, $s_{4}$, \dots{} of all the
!!{element}s of~$\Bin^\omega$. We may arrange this list, and the
elements of each sequence $s_i$ in it vertically in an array with the
positive integers on the horizontal axis, as so:
\[
\begin{array}{c|c|c|c|c}
1 & 2 & 3 & 4 & \dots \\\hline
s_{1}(1) & s_{1}(2) & s_{1}(3) & s_1(4) & \dots \\\hline
s_{2}(1)& s_{2}(2) & s_2(3) & s_2(4) & \dots \\\hline
s_{3}(1)& s_{3}(2) & s_3(3) & s_3(4) & \dots \\\hline
s_{4}(1)& s_{4}(2) & s_4(3) & s_4(4) & \dots \\\hline
\vdots & \vdots & \vdots & \vdots & \ddots
\end{array}
\]
Here $s_{1}(1)$ is a name for whatever number, a $0$ or a~$1$, is the
first member in the sequence $s_{1}$, and so on.

Now define $\overline{s}$ as follows: The $n$th member
$\overline{s}(n)$ of the sequence
$\overline{s}$ is set to
\[
\overline{s}(n) =
\begin{cases}
1 & \text{if $s_{n}(n) = 0$}\\
0 & \text{if $s_{n}(n) = 1$}.
\end{cases}
\]
In other words, $\overline{s}(n)$ has the opposite value to~$s_{n}(n)$.

Clearly $\overline{s}$ is a non-gappy infinite sequence of $0$s and
$1$s, since it is just the mirror sequence to the sequence of $0$s and
$1$s that appear on the diagonal of our array. So $\overline{s}$ is
!!a{element} of~$\Bin^\omega$. Since it is !!a{element}
of~$\Bin^\omega$, it must appear somewhere in the enumeration
of~$\Bin^\omega$, that is, $\overline{s} = s_{n}$ for some~$n$.

If $\overline{s} = s_{n}$, then for any~$m$, $\overline{s}(m) =
s_{n}(m)$. (This is just the criterion of identity for
sequences---sequences are identical when they agree at every place.)

So in particular, $\overline{s}(n) = s_{n}(n)$. $\overline{s}(n)$ must
be either an 0 or a~1. If it is a 0 then (given the definition of
$\overline{s}$) $s_{n}(n)$ must be a~$1$. But if it is a~1 then
$s_{n}(n)$ must be a~0. In either case $\overline{s}(n) \neq
s_{n}(n)$.
\end{proof}

Diagonalization need not involve the presence of an array, though the
array method is where it takes its name.

\begin{thm}
\ollabel{thm-nonenum-pownat}
$\Pow{\Int^+}$ is not enumerable.
\end{thm}

\begin{proof}
Suppose, for reductio, that $\Pow{\Int^+}$ is !!{enumerable}, and so
it has an enumeration, i.e., a list of all subsets of~$\Int^+$:
\[
Z_{1}, Z_{2}, Z_{3}, \dots
\]

We now define a set $\overline{Z}$ such that for any positive
integer~$i$, $i \in \overline{Z}$ iff $i \notin Z_{i}$:
\[
\overline{Z} = \Setabs{i \in \Int^+}{i \notin Z_i}
\]

$\overline{Z}$ is clearly a set of positive integers, and thus $\overline{Z}
\in \Pow{\Int^+}$. So $\overline{Z}$ must be $= Z_k$ for some~$k \in
\Int^+$. And if that is the case, i.e., $\overline{Z} = Z_k$, then $i
\in \overline{Z}$ iff $i \in Z_k$ for all~$i \in \Int^+$.

In particular, $k \in \overline{Z}$ iff $k \in Z_k$.

Now either $k \in Z_{k}$ or $k \notin Z_{k}$. In the first case, by
the previous line, $k \in \overline{Z}$. But we've defined
$\overline{Z}$ so that it contains exactly those~$i \in \Int^+$ which
are \emph{not} !!{element}s of~$Z_i$. So by that definition, we would
have to also have~$k \notin Z_k$. In the second case, $k \notin Z_k$.
But now $k$~satisfies the condition by which we have
defined~$\overline{Z}$, and that means that $k \in \overline{Z}$. And
as $\overline{Z} = Z_k$, we get that $k \in Z_k$ after all. Either
case leads to a contradiction.
\end{proof}

\begin{prob}
Show that $\Pow{\Nat}$ is !!{nonenumerable}.
\end{prob}

\begin{prob}
Show that the set of functions $f \colon \Int^+ \to \Int^+$ is
!!{nonenumerable} by a direct diagonal argument.
\end{prob}

\end{document}
