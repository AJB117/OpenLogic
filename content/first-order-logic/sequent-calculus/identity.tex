% Part: first-order-logic
% Chapter: sequent-calculus
% Section: equality

\documentclass[../../../include/open-logic-section]{subfiles}

\begin{document}

\olfileid{fol}{seq}{ide}

\olsection{\usetoken{P}{derivation} with \usetoken{S}{identity}}

!!^{derivation}s with the !!{identity} require additional inference rules.

\paragraph{Initial sequents for $\eq$:}

If $t$ is a closed term, then ${} \Sequent \eq[t][t]$ is an initial sequent.

\paragraph{Rules for $\eq$:}

\[
\Axiom$ \Gamma, \eq[t_1][t_2] \fCenter \Delta, !A(t_1) $
\RightLabel{$\eq$}
\UnaryInf$ \Gamma, \eq[t_1][t_2] \fCenter \Delta, !A(t_2)$
\DisplayProof
\quad
\textrm{  and  }
\quad
\Axiom$ \Gamma, \eq[t_1][t_2] \fCenter \Delta, !A(t_2) $
\RightLabel{$\eq$}
\UnaryInf$ \Gamma, \eq[t_1][t_2] \fCenter \Delta, !A(t_1)$
\DisplayProof
\]
where $t_1$ and $t_2$ are closed terms.

\begin{ex}
If $s$ and $t$ are ground terms, then $!A(s), \eq[s][t] \Proves !A(t)$:
\begin{prooftree}
\Axiom$ !A(s) \fCenter !A(s)$
\RightLabel{weak}
\UnaryInf$ !A(s), \eq[s][t] \fCenter !A(s)$
\RightLabel{$\eq$}
\UnaryInf$ !A(s), \eq[s][t] \fCenter !A(t)$
\end{prooftree}
This may be familiar as the principle of substitutability of
identicals, or Leibniz' Law.

$\Log{LK}$ proves that $\eq$ is symmetric and transitive:
\[
\Axiom$ \eq[t_1][t_2] \fCenter \eq[t_1][t_1] $
\RightLabel{weak}
\UnaryInf$ \eq[t_1][t_2] \fCenter \eq[t_1][t_1] $
\RightLabel{$\eq$}
\UnaryInf$ \eq[t_1][t_2] \fCenter \eq[t_2][t_1]$
\DisplayProof
\qquad
\Axiom$ \eq[t_1][t_2] \fCenter \eq[t_1][t_2] $
\RightLabel{weak}
\UnaryInf$ \eq[t_1][t_2], \eq[t_2][t_3] \fCenter \eq[t_1][t_2] $
\RightLabel{$\eq$}
\UnaryInf$ \eq[t_1][t_2], \eq[t_2][t_3] \fCenter \eq[t_1][t_3]$
\DisplayProof
\]
In the proof on the left, the !!{formula}~$\eq[x][t_1]$ is our $!A(x)$,
and correspondingly, $!A(t_2) \ident \eq[\Subst{x}{t_2}{x}][t_1]$. On the
right, we take $!A(x)$ to be $\eq[t_1][x]$.
\end{ex}

\begin{prob}
Give !!{derivation}s of the following sequents:
\begin{enumerate}
\item $\Sequent \lforall[x][\lforall[y][((x = y \land !A(x)) \lif !A(y))]]$
\item $\lexists[x][!A(x)] \land \lforall[y][\lforall[z][((!A(y) \land !A(z)) \lif y = z)]] \Sequent {}$\\
\hskip 1em$\lexists[x][(!A(x) \land \lforall[y][(!A(y) \lif y = x)])]$
\end{enumerate}
\end{prob}

\begin{prop}
$\Log{LK}$ with initial sequents and rules for identity is sound.
\end{prop}

\begin{proof}
Initial sequents of the form ${} \Sequent \eq[t][t]$ are valid, since
for every !!{structure}~$\Struct M$, $\Sat{M}{\eq[t][t]}$. (Note that
we assume the term $t$ to be ground, i.e., it contains no variables,
so variable assignments are irrelevant).

Suppose the last inference in a !!{derivation} is $=$. Then the
premise $\Gamma' \Sequent \Delta'$ contains $\eq[t_1][t_2]$ on the left
and $!A(t_1)$ on the right, and the conclusion is $\Gamma \Sequent
\Delta$ where $\Gamma = \Gamma'$ and $\Delta = (\Delta' \setminus
\{!A(t_1)\}) \cup \{!A(t_2)\}$. Consider a !!{structure}~$\Struct
M$. Since, by induction hypothesis, the premise $\Gamma' \Sequent
\Delta'$ is valid, either (a) for some $!E \in \Gamma'$,
$\Sat/{M}{!E}$, (b) for some $!E \in \Delta' \setminus \{!A(s)\}$,
$\Sat{M}{!E}$, or (c) $\Sat{M}{!A(t_1)}$. In both cases cases (a) and
(b), since $\Gamma = \Gamma'$, and $\Delta' \setminus \{!A(s)\}
\subseteq \Delta$, $\Struct M$ satisfies $\Gamma \Sequent \Delta$. So
assume cases (a) and (b) do not apply, but case (c) does. If (a) does
not apply, $\Sat{M}{!E}$ for all $!E \in \Gamma'$, in particular,
$\Sat{M}{\eq[t_1][t_2]}$. Therefore, $\Value{t_1}{M} = \Value{t_2}{M}$. Let
$s$ be any variable assignment, and $s'$ be the $x$-variant given by
$s'(x) = \Value{t_1}{M} = \Value{t_2}{M}$. By
\olref[fol][syn][ext]{prop:ext-formulas}, $\Sat{M}{!A(t_2)}[s]$ iff
$\Sat{M}{!A(x)}[s']$ iff $\Sat{M}{!A(t_1)}[s]$. Since $\Sat{M}{!A(t_1)}$
therefore $\Sat{M}{!A(t_2)}$.
\end{proof}

\end{document}
