% Part: first-order-logic
% Chapter: natural-deduction
% Section: soundness

\documentclass[../../../include/open-logic-section]{subfiles}

\begin{document}

\olfileid{fol}{ntd}{sou}
\olsection{Soundness}

\begin{explain}
A !!{derivation} system, such as natural deduction, is \emph{sound}
if it cannot !!{derive} things that do not actually follow.  Soundness is
thus a kind of guaranteed safety property for !!{derivation} systems.
Depending on which proof theoretic property is in question, we would
like to know for instance, that
\begin{enumerate}
\item every !!{derivable} !!{sentence} is valid;
\item if a !!{sentence} is !!{derivable} from some others, it is also a
  consequence of them;
\item if a set of !!{sentence}s is inconsistent, it is unsatisfiable.
\end{enumerate}
These are important properties of a !!{derivation} system. If any of
them do not hold, the !!{derivation} system is deficient---it would
!!{derive} too much.  Consequently, establishing the soundness of a
!!{derivation} system is of the utmost importance.
\end{explain}

\begin{thm}[Soundness]
\ollabel{thm:soundness}
If $!A$ is !!{derivable} from the !!{undischarged} assumptions
$\Gamma$, then $\Gamma \Entails !A$.
\end{thm}

\begin{proof}
\emph{Inductive Hypothesis}: The premises of an inference rule follow
from the !!{undischarged} assumptions of the subproofs ending in those premises.
 
\emph{Inductive Step}: Show that $!A$ follows from the
!!{undischarged} assumptions of the entire proof.

Let $\delta$ be a !!{derivation} of $!A$. We proceed by
induction on the number of inferences in~$\delta$.

If the number of inferences is~0, then $\delta$ consists only of an
initial !!{formula}. Every initial !!{formula} $!A$ is an
!!{undischarged} assumption, and as such, any !!{structure}
$\Struct{M}$ that satisfies all of the !!{undischarged} assumptions of
the proof also satisfies~$!A$.

If the number of inferences is greater than~0, we distinguish cases
according to the type of the lowermost inference. By induction
hypothesis, we can assume that the premises of that inference follow
from the !!{undischarged} assumptions of the sub-!!{derivation}s
ending in those premises, respectively.

First, we consider the possible inferences with only one premise.

\begin{enumerate}
\item Suppose that the last inference is \Intro{\lnot}: By inductive
  hypothesis, $\lfalse$ follows from the !!{undischarged} assumptions
  $\Gamma \cup \{!A\}$. Consider a !!{structure}~$\Struct M$. We need
  to show that, if $\Sat{M}{\Gamma}$, then $\Sat{M}{\lnot
    !A}$. Suppose for reductio that $\Sat{M}{\Gamma}$, but
  $\Sat/{M}{\lnot !A}$, i.e., $\Sat{M}{!A}$. This would mean that
  $\Sat{M}{\Gamma \cup \{!A\}}$. This is contrary to our inductive
  hypothesis. So, $\Sat{M}{\lnot !A}$.
  
\item The last inference is \Elim{\lnot}: Exercise.

\item The last inference is \Elim{\land}: There are two variants: $!A$
  or $!B$ may be inferred from the premise $!A \land !B$. Consider the
  first case.  By inductive hypothesis, $!A \land !B$ follows from the
  !!{undischarged} assumptions $\Gamma$. Consider a
  !!{structure}~$\Struct M$. We need to show that, if
  $\Sat{M}{\Gamma}$, then $\Sat{M}{!A}$. By our inductive
  hypothesis, we know that $\Sat{M}{!A\land !B}$. So, $\Sat{M}{!A}$.
  The case where $!B$ is inferred from $!A \land !B$ is handled
  similarly.
  
\item The last inference is \Intro{\lor}: There are two variants: $!A
  \lor !B$ may be inferred from the premise $!A$ or the premise
  $!B$. Consider the first case. By inductive hypothesis, $!A$ follows
  from the !!{undischarged} assumptions~$\Gamma$.  Consider a
  !!{structure}~$\Struct M$. We need to show that, if
  $\Sat{M}{\Gamma}$, then $\Sat{M}{!A \lor !B}$. Since
  $\Sat{M}{\Gamma}$, it must be the case that $\Sat{M}{!A}$, by
  inductive hypothesis. So it must also be the case that $\Sat{M}{!A \lor
    !B}$.  The case where $!A \lor !B$ is inferred from $!B$ is
  handled similarly.
  
\item The last inference is \Intro{\lif}: $!A \lif !B$ is inferred
  from a subproof with assumption~$!A$ and conclusion~$!B$. By
  inductive hypothesis, $!B$ follows from the !!{undischarged}
  assumptions $\Gamma$ and~$!A$. Consider a
  !!{structure}~$\Struct{M}$. We need to show that, if $\Gamma
  \Entails !A \lif !B$.  For reductio, suppose that for some
  !!{structure}~$\Struct{M}$, $\Sat{M}{\Gamma}$ but $\Sat/{M}{!A \lif
    !B}$. So, $\Sat{M}{!A}$ and $\Sat/{M}{!B}$. But by hypothesis,
  $!B$ is a consequence of $\Gamma \cup \{!A\}$. So, $\Sat{M}{!A \lif
    !B}$.
  
\item The last inference is \Intro{\lforall}: The premise $!A(a)$ is a
  consequence of the !!{undischarged} assumptions $\Gamma$ by
  induction hypothesis.  Consider some structure, $\Struct{M}$, such
  that $\Sat{M}{\Gamma}$. Let $\Struct{M'}$ be exactly like $\Struct{M}$
  except that $\Assign{a}{M} \neq \Assign{a}{M'}$. We must have
  $\Sat{M'}{!A(a)}$.

  We now show that $\Sat{M}{\lforall[x][!A(x)]}$. Since
  $\lforall[x][!A(x)]$ is a !!{sentence}, this means we have to show
  that for every variable assignment~$s$, $\Sat{M}{!A(x)}[s]$. Since
  $\Gamma$ consists entirely of sentences, $\Sat{M}{!B}[s]$ for all
  $!B \in \Gamma$.  Let $\Struct{M'}$ be like $\Struct{M}$ except that
  $\Assign{a}{M'} = s(x)$.  Then $\Sat{M}{!A(x)}[s]$ iff
  $\Sat{M'}{!A(a)}$ (as $!A(x)$ does not contain~$a$).  Since $a$ also
  does not occur in~$\Gamma$, $\Sat{M'}{\Gamma}$. Since $\Gamma
  \Entails !A(a)$, $\Sat{M'}{!A(a)}$. This means that
  $\Sat{M}{!A(x)}[s]$.  Since $s$ is an arbitrary variable assignment,
  $\Sat{M}{\lforall[x][!A(x)]}$.
  
\item The last inference is \Intro{\lexists}: Exercise.

\item The last inference is \Elim{\forall}: Exercise.
\end{enumerate}

Now let's consider the possible inferences with several premises:
\Elim{\lor}, \Intro{\land}, \Elim{\lif}, and \Elim{\lexists}.
\begin{enumerate}

\item The last inference is \Intro{\land}. $!A \land !B$ is inferred
  from the premises $!A$ and $!B$. By induction hypothesis, $!A$
  follows from the !!{undischarged} assumptions~$\Gamma$ and $!B$
  follows from the !!{undischarged} assumptions~$\Delta$. We have to
  show that $\Gamma \cup \Delta \Entails !A \land !B$. Consider a
  !!{structure}~$\Struct M$ with $\Sat{M}{\Gamma \cup \Delta}$. Since
  $\Sat{M}{\Gamma}$, it must be the case that $\Sat{M}{!A}$, and ince
  $\Sat{M}{\Delta}$, $\Sat{M}{!B}$, by inductive hypothesis.
  Together, $\Sat{M}{!A \land !B}$.
  
\item The last inference is \Elim{\lor}: Exercise.

\item The last inference is \Elim{\lif}. $!B$ is inferred from the
  premises $!A \lif !B$ and $!A$. By induction hypothesis, $!A \lif
  !B$ follows from the !!{undischarged} assumptions~$\Gamma$ and $!A$
  follows from the !!{undischarged} assumptions~$\Delta$. Consider a
  !!{structure}~$\Struct M$. We need to show that, if $\Sat{M}{\Gamma
    \cup \Delta}$, then $\Sat{M}{!B}$.  It must be the case that
  $\Sat{M}{!A \lif !B}$, and $\Sat{M}{!A}$, by inductive hypothesis.
  Thus it must be the case that $\Sat{M}{!B}$.
    
\item The last inference is \Elim{\lexists}: Exercise.
\end{enumerate}
\end{proof}

\begin{prob}
Complete the proof of \olref[fol][ntd][sou]{thm:soundness}.
\end{prob}

\begin{cor}
\ollabel{cor:weak-soundness}
If $\Proves !A$, then $!A$ is valid.
\end{cor}

\begin{cor}
\ollabel{cor:consistency-soundness}
If $\Gamma$ is satisfiable, then it is consistent.
\end{cor}

\begin{proof}
We prove the contrapositive.  Suppose that $\Gamma$ is not consistent.
Then $\Gamma \Proves \lfalse$, i.e., there is !!a{derivation} of
$\lfalse$ from !!{undischarged} assumptions in~$\Gamma$. By
\olref{thm:soundness}, any !!{structure} $\Struct{M}$ that satisfies
$\Gamma$ must satisfy $\lfalse$.  Since $\Sat/{M}{\lfalse}$ for every
!!{structure}~$\Struct{M}$, no $\Struct{M}$ can satisfy $\Gamma$,
i.e., $\Gamma$ is not satisfiable.
\end{proof}

\end{document}
