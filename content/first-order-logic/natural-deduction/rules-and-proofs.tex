% Part: first-order-logic
% Chapter: natural-deduction
% Section: rules-and-proofs

\documentclass[../../../include/open-logic-section]{subfiles}

\begin{document}

\olfileid{fol}{ntd}{rul}

\olsection{Rules and \usetoken{P}{derivation}}

Let $\Lang L$ be a first-order language with the usual constants,
!!{variable}s, logical symbols, and auxiliary symbols (parentheses
and the comma).

\begin{defn}[Inference]
An \emph{inference} is an expression of the form
\[
\AxiomC{$!A$}
\UnaryInfC{$!C$}
\DisplayProof
\quad
\textrm{  or  }
\quad
\AxiomC{$!A$}
\AxiomC{$!B$}
\BinaryInfC{$!C$}
\DisplayProof
\]
where $!A, !B$, and $!C$ are !!{formula}s. $!A$ and $!B$ are called
the \emph{upper !!{formula}s} or \emph{premises} and $!C$ the
\emph{lower !!{formula}s} or \emph{conclusion} of the inference.
\end{defn}

The rules for natural deduction are divided into two main types:
\emph{propositional} rules (quantifier-free) and \emph{quantifier}
rules.  The rules come in pairs, an introduction and an elimination
rule for each !!{operator}. They introduced an !!{operator} in the
conclusion or remove and !!{operator} from a premise of the rule.
Some of the rules allow an assumption of a certain type to be
\emph{!!{discharged}}. To indicate which assumption is !!{discharged}
by which inference, we also assign labels to both the assumption and
the inference.  This is indicated by writing the assumption formula as
``$\Discharge{!A}{n}$''.

It is customary to consider rules for all !!{operator}s, even for
those (if any) that we consider as defined.

\subsection{Propositional Rules}

\paragraph{Rules for $\lfalse$}

\[
\AxiomC{$!A$}
\AxiomC{$\lnot !A$}
\RightLabel{\Intro{\lfalse}}
\BinaryInfC{$\lfalse$}
\DisplayProof
\quad
\AxiomC{$\lfalse$}
\RightLabel{\Elim{\lfalse}}
\UnaryInfC{$!A$}
\DisplayProof
\]

\paragraph{Rules for $\land$}

\[
\AxiomC{$!A$}
\AxiomC{$!B$}
\RightLabel{\Intro{\land}}
\BinaryInfC{$!A \land !B$}
\DisplayProof
\quad
\AxiomC{$!A \land !B$}
\RightLabel{\Elim{\land}}
\UnaryInfC{$!A$}
\DisplayProof
\quad
\AxiomC{$!A \land !B$}
\RightLabel{\Elim{\land}}
\UnaryInfC{$!B$}
\DisplayProof
\]

\paragraph{Rules for $\lor$}

\[
\AxiomC{$!A$}
\RightLabel{\Intro{\lor}}
\UnaryInfC{$!A \lor !B$}
\DisplayProof
\quad
\AxiomC{$!B$}
\RightLabel{\Intro{\lor}}
\UnaryInfC{$!A \lor !B$}
\DisplayProof
\quad
\AxiomC{$!A \lor !B$}
\AxiomC{$\Discharge{!A}{n}$}
\DeduceC{$!C$}
\AxiomC{$\Discharge{!B}{n}$}
\DeduceC{$!C$}
\DischargeRule{\Elim{\lor}}{n}
\TrinaryInfC{$!C$}
\DisplayProof
\]

\paragraph{Rules for $\lnot$}

\[
\AxiomC{$\Discharge{!A}{n}$}
\noLine
\DeduceC{$\lfalse$}
\DischargeRule{\Intro{\lnot}}{n}
\UnaryInfC{$\lnot !A$}
\DisplayProof
\quad
\AxiomC{$\lnot \lnot !A$}
\RightLabel{\Elim{\lnot}}
\UnaryInfC{$!A$}
\DisplayProof
\]

\paragraph{Rules for $\lif$}

\[
\AxiomC{$\Discharge{!A}{n}$}
\DeduceC{$!B$}
\DischargeRule{\Intro{\lif}}{n}
\UnaryInfC{$!A \lif !B$}
\DisplayProof
\quad
\AxiomC{$!A$}
\AxiomC{$!A \lif !B$}
\RightLabel{\Elim{\lif}}
\BinaryInfC{$!B$}
\DisplayProof
\]

\subsection{Quantifier Rules}

\paragraph{Rules for $\lforall$}

\[
\AxiomC{$!A(a)$}
\RightLabel{\Intro{\lforall}}
\UnaryInfC{$\lforall[x][!A(x)]$}
\DisplayProof
\quad
\AxiomC{$\lforall[x][!A(x)]$}
\RightLabel{\Elim{\lforall}}
\UnaryInfC{$!A(t)$}
\DisplayProof
\]
where $t$ is a ground term, and $a$ is a constant which does not occur
in $!A$, or in any assumption which is !!{undischarged} in the
!!{derivation} ending with the premise~$!A$. We call $a$ the
\emph{eigenvariable} of the \Intro{\forall} inference.

\paragraph{Rules for $\lexists$}

\[
\AxiomC{$\Atom{!A}{a}$}
\RightLabel{\Intro{\lexists}}
\UnaryInfC{$\lexists[x][\Atom{!A}{x}]$}
\DisplayProof
\quad
\AxiomC{$\lexists[x][\Atom{!A}{x}]$}
\AxiomC{[$\Atom{!A}{a}$]$^n$}
\DeduceC{$!C$}
\RightLabel{$\lexists$Elim$_n$}
\BinaryInfC{$!C$}
\DisplayProof
\]
where $t$ is a ground term, and $a$ is a constant which does not occur
in the premise $\lexists[x][!A(x)]$, in $!C$, or any assumption which
is !!{undischarged} in the !!{derivation}s ending with the two
premises~$!C$ (other than the assumptions $!A(a)$).  We call $a$ the
\emph{eigenvariable} of the \Elim{\lexists} inference.

The condition that an eigenvariable not occur in the upper sequent of
the $\lforall$ intro or $\lexists$ elim inference is called the
\emph{eigenvariable condition}.

\begin{explain}
We use the term ``eigenvariable'' even though $a$ in the above rules
is a constant. This has historical reasons.

In \Intro{\lexists} and \Elim{\lforall} there are no restrictions, and
the term~$t$ can be anything, so we do not have to worry about any
conditions. However, because the $t$ may appear elsewhere in the
!!{derivation}, the values of~$t$ for which the !!{formula} is satisfied are
constrained. On the other hand, in the \Elim{\lexists} and $\lforall$
intro rules, the eigenvariable condition requires that $a$ does not
occur anywhere else in the formula. Thus, if the upper !!{formula} is
valid, the truth values of the formulas other than $\Atom{!A}{a}$ are
independent of~$a$.
\end{explain}

\begin{explain}
Natural deduction systems are meant to closely parallel the informal
reasoning used in mathematical proof (hence it is somewhat
``natural''). Natural deduction proofs begin with assumptions.
Inference rules are then applied. Assumptions are ``!!{discharged}''
by the \Intro{\lnot}, \Intro{\lif}, \Elim{\lor} and \Elim{\lexists}
inference rules, and the label of the !!{discharged} assumption is
placed beside the inference for clarity.
\end{explain}

\begin{defn}[Initial !!^{formula}]
An \emph{initial !!{formula}} or \emph{assumption} is any !!{formula}
in the topmost position of any branch.
\end{defn}

\begin{defn}[!!^{derivation}]
A \emph{!!{derivation}} of !!a{formula} $!A$ from assumptions~$\Gamma$
is a tree of !!{formula}s satisfying the following conditions:
\begin{enumerate}
\item The topmost !!{formula}s of the tree are either in $\Gamma$ or
  are !!{discharged} by an inference in the tree.
\item Every !!{formula} in the tree is an upper !!{formula} of an
  inference whose lower !!{formula} stands directly below that !!{formula} in
  the tree.
\end{enumerate}
We then say that $!A$ is the \emph{end-!!{formula}} of the
!!{derivation} and that $!A$ is \emph{!!{derivable}} from~$\Gamma$.
\end{defn}

\begin{defn}[Theorem]
!!^a{sentence} $!A$ is a \emph{theorem} if it is !!{derivable}
from the empty set.
\end{defn}

\end{document}
