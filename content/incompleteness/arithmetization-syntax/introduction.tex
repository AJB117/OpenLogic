% Part: incompleteness
% Chapter: arithmetization-syntax
% Section: introduction

\documentclass[../../../include/open-logic-section]{subfiles}

\begin{document}

\olfileid{inc}{art}{int}
\olsection{Introduction}

In order to connect computability and logic, we need a way to talk
about the objects of logic (symbols, terms, !!{formula}s,
!!{derivation}s), operations on them, and their properties and
relations, in a way amenable to computational treatment.  We can do
this directly, by considering computable functions and relations on
symbols, sequences of symbols, and other objects built from them.
Since the objects of logical syntax are all finite and built from
!!a{enumerable} sets of symbols, this is possible for some models of
computation.  But other models of computation are restricted to
numbers, their relations and functions.  Moreover, ultimately we also
want to deal with syntax in certain theories, specifically, in
theories formulated in the language of arithmetic.  In these cases it
is necessary to \emph{arithmetize} syntax, i.e., to represent
syntactic objects, operations, and relations as numbers, arithmetical
functions, and arithmetical relations, respectively.  This is done by
assigning numbers to symbols as their ``codes.'' Since we can deal
with sequences of numbers purely arithmetically by the
powers-of-primes coding, we can extend this coding of individual
symbols to coding of sequences of symbols (such as terms and
!!{formula}s) and also arrangements of such sequences (such as
!!{derivation}s).  This extended coding is called ``G\"odel
numbering.''  Because the sequences of interest (terms, !!{formula}s,
!!{derivation}s) are inductively defined, and the operations and
relations on them are computable, the corresponding sets, operations,
and relations are in fact all computable, and almost all of them are
in fact primitive recursive.

\end{document}
