% Part: first-order-logic
% Chapter: model-theory
% Section: nonstandard-arithmetic

\documentclass[../../../include/open-logic-section]{subfiles}

\begin{document}

\olfileid{mod}{bas}{nsa}
\section{Non-standard Models of Arithmetic}

\begin{defn}
Let $\Lang{L_N}$ be the !!{language} of arithmetic, comprising a
!!{constant} $\Obj{0}$, a 2-place !!{predicate} $<$, a 1-place
!!{function} $\prime$, and 2-place !!{function}s $+$ and $\times$.
\begin{enumerate}
\item The \emph{standard model} of arithmetic is the !!{structure}
  $\Struct{N}$ for $\Lang{L_N}$ having $\Nat = \{ 0, 1, 2, \dots\}$
  and interpreting $\Obj{0}$ as $0$, $<$ as the less-than relation
  over~$\Nat$, and $\prime$, $+$ and $\times$ as successor, addition,
  and multiplication over $\Nat$, respectively.
\item \emph{True arithmetic} is the theory $\Theory{N}$.
\end{enumerate}
\end{defn}

When working in $\Lang{L_N}$ we abbreviate each term of
the form $\Obj{0}^{\prime\dots\prime}$, with $n$ applications of the
successor function to $\Obj{0}$, as~$\num{n}$.

\begin{defn}
A !!{structure} $\Struct{M}$ for $\Lang{L_N}$ is \emph{standard} if
and only $\Struct{N} \iso \Struct{M}$.
\end{defn}

\begin{thm}\ollabel{thm:non-std}
There are non-standard !!{enumerable} models of true arithmetic.
\end{thm}

\begin{proof}
Expand $\Lang{L_N}$ by introducing a new !!{constant}~$c$, and
consider the theory
\[
\Theory{N} \cup \Setabs{\num{n} < c}{n \in \Nat}.
\]
The theory is finitely satisfiable, so by compactness it has a model
$\Struct{M}$, which can be taken to be !!{enumerable} by the Downward
L\"owenheim-Skolem theorem. Where $\Domain{M}$ is the domain of
$\Struct{M}$, let $\Struct{M}$ interpret the non-logical constants of
$\Lang{L}$ as $\mathbf{z} = \Assign{\Obj{0}}{M} \in \Domain{M}$, ${\prec} =
\Assign{<}{M} \subseteq M^2$, $* = \Assign{\prime}{M} \colon
\Domain{M} \to \Domain{M}$, and $\oplus = \Assign{+}{M}, \otimes =
\Assign{\times}{M} \colon \Domain{M}^2 \to \Domain{M}$. For each $x
\in \Domain{M}$, we write $x^*$ for the element of $\Domain{M}$
obtained from $x$ by application of~$*$.

Now, if $h$ were an isomorphism of $\Struct{N}$ and $\Struct{M}$,
there would be $n \in \Nat$ such that $h(n) = \Assign{c}{M}$.  So let
$s$ be any assignment in $\Struct{N}$ such that $s(x) = n$. Then
$\Sat{N}{\eq[\num{n}][x]}[s]$; by the proof of \olref[iso]{thm:isom},
also $\Sat{M}{\eq[\num{n}][x]}[h \circ s]$, so that $\Assign{c}{M} =
\mathbf{z}^{*\cdots *}$ (with $*$ iterated $n$ times). But this is impossible
since by assumption $\Sat{M}{\num{n} < c}$ and $\prec$ is
irreflexive. So $\Struct{M}$ is non-standard.
\end{proof}

\begin{prob}
A relation $R$ over a set $X$ is \emph{well-founded} if and only if
there are no infinite descending chains in~$R$, i.e., if there are no
$x_0$, $x_1$, $x_2$,~\dots in $X$ such that $\dots x_2Rx_1Rx_0$.  Assuming
Zermelo-Fraenkel set theory~$ZF$ is consistent, show that there are
non-well-founded models of $ZF$, i.e., models $\mathfrak{M}$ such that
$\dots x_2 \in x_1 \in x_0$.
\end{prob}

Since the non-standard model $\Struct{M}$ from \olref{thm:non-std} is
elementarily equivalent to the standard one, a number of properties of
$\Struct{M}$ can be derived. The rest of this section is devoted to
such a task, which will allow us to obtain a precise characterization
of !!{enumerable} non-standard models of $\Theory{N}$.

\begin{enumerate}
\item No member of $\Domain{M}$ is $\prec$-less than itself: the sentence
  $\lforall[x][\lnot x < x]$ is true in $\Struct{N}$ and therefore in
  $\Struct{M}$.
\item By a similar reasoning we obtain that $\prec$ is a \emph{linear
    ordering} of $\Domain{M}$, i.e., a total, irreflexive, transitive relation
  on $\Domain{M}$. 
\item The element $\mathbf{z}$ is the $\prec$-least element of $\Domain{M}$.
\item Any member of $\Domain{M}$ is $\prec$-less than its $*$-successor and
  $x^*$ is the $\prec$-least member of $\Domain{M}$ greater than $x$.
\item $\Struct{M}$ contains an initial segment (of $\prec$) isomorphic
  to $\Nat$: $\mathbf{z}, \mathbf{z}^*, \mathbf{z}^{**}, \dots$, which
  we call the \emph{standard part} of $\Domain{M}$. Any other member
  of $\Domain{M}$ is \emph{non-standard}. There must be non-standard
  members of $\Domain{M}$, or else the function $h$ from the proof of
  \olref{thm:non-std} is an isomorphism.  We use $n, m, \dots$ as
  !!{variable}s ranging on this standard part of $\Struct{M}$.
\item Every non-standard element is greater than any standard one;
  this is because for every $n \in \Nat$,
  \[
  \Sat{N}{\lforall[z][(\lnot(\eq[z][\Obj{0}] \lor
  \dots \lor \eq[z][\num{n}]) \lif \num{n} < z)]},
  \]
  so if $z \in \Domain{M}$ is different from all the standard
  elements, it must be \emph{greater} than all of them.
\item Any member of $\Domain{M}$ other than $\mathbf{z}$ is the
  $*$-successor of some unique element of $\Domain{M}$, denoted by
  $^*x$. If $x = y^*$ then both $x$ and $y$ are standard if one of
  them is (and both non-standard if one of them is).
\item Define an equivalence relation $\approx$ over $\Domain{M}$ by
  saying that $x \approx y$ if and only if for some \emph{standard}
  $n$, either $x \oplus n = y$ or $y \oplus n =x$. In other words, $x
  \approx y$ if and only if $x$ and $y$ are a finite distance
  apart. If $n$ and $m$ are standard then $n \approx m$. Define the
  \emph{block} of $x$ to be the equivalence class $[x] = \Setabs{y}{x
  \approx y}$.
\item Suppose that $x \prec y$ where $x \not\approx y$. Since
  $\Sat{N}{\lforall[x][\lforall[y][(x < y \lif (x' < y \lor x' =
      y))]]}$, either $x^* \prec y$ or $x^* = y$. The latter is
  impossible because it implies $x \approx y$, so $x \prec
  y$. Similarly, if $x \prec y$ and $x \not\approx y$, then $x \prec
  {^*y}$. Therefore if $x \prec y$ and $x \not\approx y$, then every
  $w \approx x$ is $\prec$-less than every $v \approx y$. Accordingly,
  each block $[x]$ forms a doubly infinite chain
  \[
  \cdots \prec  {^{**}x} \prec {^*}x \prec x \prec x^* \prec x^{**}
  \prec \cdots
  \]
  which is referred to as a $Z$-chain because it has the order type of
  the integers.
\item The $\prec$ ordering can be lifted up to the blocks: if $x \prec y$
  then the block of $x$ is less than the block of $y$. A block is
  \emph{non-standard} if it contains a non-standard element. The
  standard block is the least block. 
\item There is no least non-standard block: if $y$ is non-standard
  then there is a $x \prec y$ where $x$ is also non-standard and $x
  \not\approx y$. Proof: in the standard model $\Struct{N}$, every
  number is divisible by two, possibly with remainder one, i.e.,
  $\Sat{N}{\lforall[y][\lforall[x][(y = x + x \lor y = x + x +
        \Obj{0}')]]}$. By elementary equivalence, for every $y \in
  \Domain{M}$ there is $x \in \Domain{M}$ such that either $x \oplus x
  = y$ or $x \oplus x \oplus \mathbf{z}^*= y$. If $x$ were standard,
  then so would be $y$; so $x$ is non-standard. Furthermore, $x$ and
  $y$ belong to different blocks, i.e, $x \not\approx y$.  To see
  this, assume they did belong to the same block, i.e., $x \oplus n =
  y$ for some standard $n$. If $y = x \oplus x$, then $x \oplus n = x
  \oplus x$, whence $x = n$ by the cancellation law for addition
  (which holds in $\Struct{N}$ and therefore in $\Struct{M}$ as well),
  and $x$ would be standard after all. Similarly if $y = x \oplus x
  \oplus \mathbf{z}^*$.
\item By a similar argument, there is no greatest block. 
\item The ordering of the blocks is dense: if $[x]$ is less than $[y]$
  (where $x \not\approx y$), then there is a block $[z]$ distinct from
  both that is between them. Suppose $x \prec y$. As before, $x \oplus
  y$ is divisible by two (possibly with remainder) so there is a $u
  \in \Domain{M}$ such that either $x \oplus y = u \oplus u$ or $x
  \oplus y = u \oplus u \oplus \mathbf{z}^*$. The element $u$ is the
  average of $x$ and $y$, and so is between them. Assume $x \oplus y =
  u \oplus u$ (the other case being similar): if $u \approx x$ then
  for some standard $n$:
  \[
  x \oplus y = x \oplus n \oplus x \oplus n,
  \]
  so $y = x \oplus n \oplus n$ and we would have $x \approx y$,
  against assumption. We conclude that $u \not\approx x$. A similar
  argument gives $u \not\approx y$.
\end{enumerate}
The non-standard blocks are therefore ordered like the rationals: they
form !!a{enumerable} linear ordering without endpoints.  It follows
that for any two !!{enumerable} non-standard models $\Struct{M}_1$ and
$\Struct{M_2}$ of true arithmetic, their reducts to the language
containing $<$ and $=$ only are isomorphic. Indeed, an isomorphism $h$
can be defined as follows: the standard parts of $\Struct{M_1}$ and
$\Struct{M_2}$ are isomorphic to the standard model $\Struct{N}$ and
hence to each other. The blocks making up the non-standard part are
themselves ordered like the rationals and therefore by
\olref[dlo]{thm:cantorQ} are isomorphic; an isomorphism of the blocks
can be extended to an isomorphism \emph{within} the blocks by matching
up arbitrary elements in each, and then taking the image of the
successor of $x$ in $\Struct{M_1}$ to be the successor of the image of
$x$ in $\Struct{M_2}$. Note that it does \emph{not} follow that
$\mathfrak{M}_1$ and $\mathfrak{M}_2$ are isomorphic in the full
language of arithmetic (indeed, isomorphism is always relative to a
signature), as there are non-isomorphic ways to define addition and
multiplication over $\Domain{M_1}$ and $\Domain{M_2}$. (This also
follows from a famous theorem due to Vaught that the number of
countable models of a complete theory cannot be~2.)

\begin{prob}
Show that there can be no greatest block in a non-standard model of
arithmetic.
\end{prob}

\begin{prob} 
Let $\Lang{L}$ be the first-order !!{language} containing $<$ as its
only !!{predicate} (besides $\eq$), and let $\Struct{N} = (\Nat,
<)$. All the finite or cofinite subsets of $\Struct{N}$ are
definable. Show that these are the \emph{only} definable subsets of
$\Struct{N}$.
 
(Hint: First, let $\Obj{prc}(x,y)$ be the
$\Lang{L}$-formula abbreviating ``$x$ is the immediate predecessor
of $y$:''
\[
x<y \land \lnot \lexists[z][(x<z \land z < y)].
\]
Now, to any definable subset of $\Struct{N}$ there corresponds a
formula $!A(x)$ in $\Lang{L}$. For any such $!A$,
consider the sentence $!D$:
\[
\lexists[x][\lforall[y][\lforall[z][((x<y \land x<z \land \Obj{prc}(y,z)
  \land !A(y)) \lif !A(z))]]].
\]
Show that $\Sat{N}{!D}$ if and only if the subset of
$\Struct{N}$ defined by $!A$ is either finite or cofinite.

Now, let $\Struct{M}$ be a non-standard model elementarily
equivalent to $\Struct{N}$.  If $a \in \Domain{M}$ is
non-standard, let $b, c \in \Domain{M}$ be greater than $a$, and
let $b$ be the immediate predecessor of~$c$. Then there is an
automorphism $h$ of $\Domain{M}$ such that $h(b)=c$
(why?). Therefore, if $b$ satisfies $!A$, so does $c$ (why?). It
follows that $!D$ is true in $\Struct{M}$, and hence also in
$\Struct{N}$. But this implies that the subset of $\Struct{N}$
defined by $!A$ is either finite or co-finite.
\end{prob}

\end{document}
