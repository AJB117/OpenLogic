% Part: methods
% Chapter: methods 
% Section: argument-forms

\documentclass[../../../include/open-logic-section]{subfiles}

\begin{document}

\olfileid{mth}{mth}{arg}
\olsection{Forms of Argument}

Reduction is a common argumentative strategy. It can be positive or
negative.

A positive reduction shows that a problem is solvable given a solution to a
known solvable problem. In this case, one knows that problem $p$ is
solvable, proves that if $p$ is solvable problem $q$ is solvable, and
concludes $q$ is solvable.

A negative one shows that any solution to a problem also solves a known to
be unsolvable problem. In this case one knows that problem $q$ is
unsolvable, proves that if $p$ is solvable then $q$ is solvable, and
concludes that $p$ is unsolvable.

Both forms of diagonalization are reductios. Reductio is short for reductio
ad absurdum, which roughly translated means \emph{reduce to an absurdity}.
A reductio works by assuming the opposite of what you want show, and then
arguing to a contradiction.

QED abbreviates quod est demonstratum---that is demonstrated---and is
traditionally used to mark the end of a non-reductio proof.

QEA abbreviates quod est absurdum---that is absurd---and is traditionally
used to mark the end of a reductio.

Modern authors sometimes use a box---$\Box$---for both purposes.

The reduction proof is only incidentally presented as a reductio---it can
be structured as a direct proof using modus tollens, but at the cost of
length.

\end{document}
