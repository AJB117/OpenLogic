% Part: methods
% Chapter: methods 
% Section: axiomatic-deduction

\documentclass[../../include/open-logic-section]{subfiles}

\begin{document}

\olfileid{mth}{mth}{axd}
\olsection{Axiomatic Deduction}

\begin{explain}

    The derived rules are generally useful for simplifying the problem
    you have to deal with. Look at the form of the problem you're dealing
with, and try to match it to the available rules. Ask yourself what you
would have to show in order to prove something of that form.

    For instance, if you need to show something of the form $\Sigma
\Proves \lforall[x][!A]$ (with $x$ not free in $\Sigma$) then you might
want to work backwards using the generalization theorem, and say that
    it would suffice to show $\Sigma \Proves !A$.

    In order to avoid confusing yourself, work backwards to reduce the
    problem to the simplest form possible, and then solve that problem.

    Intuitively, think of it like being given directions to find a place.
To get to my office, it suffices to find the Philosophy Department main
office. To find the main office, it suffices to find the 3rd floor of
    Clearihue B wing. To find that floor, it suffices to find the B
    building. To find that building, it suffices to find the university
    centre. And if you know how to get there, then you're done.

    This is just the same pattern that you follow when you're working
    backwards in a !!{derivation}.

    But there are generally several different !!{derivation}s of any given
    !!{formula}, so there won't necessarily be a unique place when you stop
    working backwards. Practice will help you.
\end{explain}

\begin{ex}
We can show $\lexists[x]\lforall[y][R(x,y)] \Proves
\lforall[y]\lexists[x][R(x,y)]$.

First, it suffices to show that $\lexists[x]\lforall[y][R(x,y)] \Proves
\lexists[x][R(x,y)]$, by the generalisation theorem. Then we can fill
    in equivalences:\\
$\lnot \lforall[x] \lnot \lforall[y][R(x,y)] \Proves \lnot
\lforall[x][\lnot R(x,y)]$.

And then we can apply contraposition, so it suffices to show:\\ 
$\lforall[x][\lnot R(x,y)] \Proves \lforall[x]\lnot \lforall[y][R(x,y)]$. 

Then by generalisation, again, it suffices to show:\\
$\lforall[x][\lnot R(x,y)] \Proves \lnot \lforall[y][R(x,y)]$. 

But then by reductio, we can just show that $\Sigma = \{\lforall[x][\lnot
R(x,y)], \lforall[y][R(x,y)]
\}$ is inconsistent by deriving a contradiction from that set. This is
whatwe do below.

First, $\Proves \lforall[x][\lnot R(x,y)] \rightarrow \lnot R(x,y)$ by
axiom
group one, so $\lforall[x][\lnot R(x,y)] \Proves \lnot R(x,y)$. And for thesame
reason, we have $\lforall[y][ R(x,y)] \Proves R(x,y)$. So we have shown
that the
set is inconsistent. So we are done.
\end{ex}

\end{document}